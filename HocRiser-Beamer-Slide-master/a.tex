\documentclass{beamer}
\usepackage{ctex, hyperref}
\usepackage[T1]{fontenc}
\usepackage{latexsym,amsmath,xcolor,multicol,booktabs,calligra}
\usepackage{graphicx,pstricks,listings,stackengine}
\def\cmd#1{\texttt{\color{red}\footnotesize $\backslash$#1}}
\def\env#1{\texttt{\color{blue}\footnotesize #1}}
\definecolor{deepblue}{rgb}{0,0,0.5}
\definecolor{deepred}{rgb}{0.6,0,0}
\definecolor{deepgreen}{rgb}{0,0.5,0}
\definecolor{halfgray}{gray}{0.55}
\lstset{
	basicstyle=\ttfamily\small,
	keywordstyle=\bfseries\color{deepblue},
	emphstyle=\ttfamily\color{deepred},
	stringstyle=\color{deepgreen},
	numbers=left,
	numberstyle=\small\color{halfgray},
	rulesepcolor=\color{red!20!green!20!blue!20},
	frame=shadowbox,
}

%---------------------------------------------------

\author{HocRiser}
\title{数学分析}
\subtitle{大一上期末总结}
\institute{吉林大学 20级唐计}
\date{2020年12月26日}
\usepackage{waseda}

\begin{document}

\kaishu
\begin{frame}
	\titlepage
	\begin{figure}[htpb]
		\begin{center}
			\includegraphics[width=0.2\linewidth]{pic/HocRiser.jpg}
		\end{center}
	\end{figure}/Users/cloudsky/Documents/应用/LaTeX/HocRiser-Beamer-Slide-master/pic/HocRiser.jpg
\end{frame}

\begin{frame}
	\tableofcontents[sectionstyle=show,subsectionstyle=show/shaded/hide,subsubsectionstyle=show/shaded/hide]
\end{frame}

%---------------------------------------------------

\section{数列极限}

\subsection{数列极限的定义与基本性质}

\begin{frame}{数列极限的定义}
	\begin{itemize}[<+-| alert@+>] 
		\item 设$\{x_n\}$为一个数列,$A$为一个给定实数,如果对于任意给定的正数$\varepsilon$,都存在正整数$N$,使得当$n>N$时,就有$$|x_n-A|<\varepsilon$$
	\end{itemize}
\end{frame}

\begin{frame}
	\begin{center}
		{\Huge\calligra Thanks!}
	\end{center}
\end{frame}

%---------------------------------------------------

\end{document}